\documentclass{article}
\usepackage{graphicx}
\usepackage{amsmath}
\usepackage{pgfplots}
\usepackage{physics}
\usepackage{cancel}
\usepackage{enumitem}
\usepackage{txfonts}

\pgfplotsset{compat=1.18}

\usepackage[a4paper, top=1cm, bottom=2cm, left=2cm, right=2cm, includehead, includefoot]{geometry}

\begin{document}

\noindent
Math 110B - Calculus II  \hfill Prof. Jamey Bass

\noindent\rule{\textwidth}{0.4pt}

\begin{center}
    \textbf{\LARGE Homework 12} \\
    \vspace{12pt}
    \large Aaron W. Tarajos \\
    \textit{\today}
\end{center}

\noindent\rule{\textwidth}{0.4pt}

\section*{11.10 Question 25}
 Find the Taylor series for $f(x)$ centered at the given value
of $a$. [Assume that $f$ has a power series expansion. Do not show
that $R_n(x)\to 0$]. Also find the associated radius of convergence.

\[
	f(x) = \sin x, \quad a = \pi
\]

\subsection*{Solution}
Firs we write out the first couple of derivates for $f(x)$ and solve for $x = \pi$
\begin{align*}
	f^\prime(x) &= \cos x \\
	f^{\prime\prime}(x) &= -\sin x \\
	f^{\prime\prime\prime}(x) &= -\cos x \\
	f^{(4)}(x) &= \sin x
\end{align*}
and we see that only the odd terms will construct the series because even derivatives evaluated at $\pi$ are zero; so we can write the odd terms of the taylor series as;
\begin{align*}
	n = 1 &\rightarrow \frac{1}{1!}(x-\pi) \\
	n=3 &\rightarrow \frac{1}{3!}(x-\pi)^3 \\
	n=5 &\rightarrow \frac{-1}{5!}(x-\pi)^5
\end{align*}
which we use to construct the general term of the series;
\begin{equation}
	f(x) = -\sum_{n=0}^\infty \frac{(-1)^n}{(2n+1)!}(x-\pi)^{2n+1}
\end{equation}
and then find the radius of convergene using the ration test;
\begin{align*}
	L &= \lim_{n \to \infty} \left| \frac{a_{n+1}}{a_n} \right| \\
	  &= \lim_{n \to \infty} \left| \frac{(x - \pi)^{2(n+1)+1}/(2(n+1)+1)!}{(x - \pi)^{2n+1}/(2n+1)!} \right| \\
	  &= \lim_{n \to \infty} \left| \frac{(x - \pi)^2}{(2n+3)(2n+2)} \right| = 0
\end{align*}
because $L=0$ the series converges for all $x \in \mathbb{R}$


\end{document}
