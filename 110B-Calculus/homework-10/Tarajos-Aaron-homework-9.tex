\documentclass{article}
\usepackage{graphicx}
\usepackage{amsmath}
\usepackage{pgfplots}
\usepackage{physics}
\usepackage{cancel}
\usepackage{enumitem}
\usepackage{txfonts}

\pgfplotsset{compat=1.18}

\usepackage[a4paper, top=1cm, bottom=2cm, left=2cm, right=2cm, includehead, includefoot]{geometry}

\begin{document}

\noindent
Math 110B - Calculus II  \hfill Prof. Jamey Bass

\noindent\rule{\textwidth}{0.4pt}

\begin{center}
    \textbf{\LARGE Homework 9} \\
    \vspace{12pt}
    \large Aaron W. Tarajos \\
    \textit{\today}
\end{center}

\noindent\rule{\textwidth}{0.4pt}

\section*{11.5 Question 7}
Test the series for convergence or divergence.

\[
	\sum_{n=1}^\infty = (-1)^n \frac{3n-1}{2n+1}
\]

\subsection*{Solution}
Leibnitz rule for alternating series tells us that the series will converge if the series is monotonically decreasing towards zero then the series is convergent. We can find the limit relatively easily;
\begin{align*}
	\lim_{n \to \infty} \frac{3n-1}{2n+1} &= \lim_{n \to \infty} \frac{n\left( 3 - \frac{1}{n}\right)}{n \left( 2 + \frac{1}{n} \right)} \\
					      &= \lim_{n \to \infty} \frac{\left( 3 - \frac{1}{n}\right)}{\left( 2 + \frac{1}{n} \right)} \\
					      &= \frac{3}{2}
\end{align*}
By Leibnitz rule, the series does not converge because it does not approach zero in the limit.

\end{document}
